\documentclass{article}

\usepackage[ngerman]{babel}
\usepackage[utf8]{inputenc}
\usepackage[T1]{fontenc}
\usepackage{hyperref}
\usepackage{csquotes}

\usepackage[
    backend=biber,
    style=apa,
    sortlocale=de_DE,
    natbib=true,
    url=false,
    doi=false,
    sortcites=true,
    sorting=nyt,
    isbn=false,
    hyperref=true,
    backref=false,
    giveninits=false,
    eprint=false]{biblatex}
\addbibresource{../references/bibliography.bib}

\title{Review des Papers "Ethik im Umgang mit Daten" von Jelena Jovanoski}
\author{Sara Broch Jahandar}
\date{03.06.2024}

\begin{document}
\parindent=0em
\parskip=0.5em
\maketitle

\section{Formales}
Das Projekt wurde mehrheitlich in einem einfachen Deutsch geschrieben. Es gibt nur ein paar Begriffe, die zum besseren Verständnis 
hätten geklärt werden müssen. (zB. Datensätze, gelabelte Daten).
Es gibt viele Schreibfehler und unlogische Sätze. Dies hätte man durch durchlesen 
vermeiden können.
Mit diesen zwei Verbesserungsvorschlägen können Leute, die sich mit dem Thema noch nicht so gut auskennen mehr aus de Projekt
mitnehmen.

\section{Inhalt}
Ich finde, dass die Arbeit das Thema nicht zu tief aber auch nicht oberflächlich behandelt.
Bei der Erklärung des überwachten und unüberwachten Lernens wird etwas nicht erklärt : gelabelte Trainingsdaten (Daten, wo die Lösung vorgegeben ist)
werden beim Trainingsprozess nur beim überwachten Lernen verwendet. \citep{bigdatainsider}
Der restliche Inhalt stimmt mit dem was ich gelernt habe überein. Ein paar Inhalte wie Gefahren habe ich in meinem Projekt nicht behandelt
und kann deshalb keine Inhaltskritik geben.
\printbibliography

\end{document}
