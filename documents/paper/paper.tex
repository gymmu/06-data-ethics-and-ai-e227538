\documentclass{report}

\usepackage[ngerman]{babel}
\usepackage[utf8]{inputenc}
\usepackage[T1]{fontenc}
\usepackage{blue}
\usepackage{hyperref}
\usepackage{csquotes}
\usepackage[a4paper]{geometry}

\usepackage[
    backend=biber,
    style=apa,
    sortlocale=de_DE,
    natbib=true,
    url=false,
    doi=false,
    sortcites=true,
    sorting=nyt,
    isbn=false,
    hyperref=true,
    backref=false,
    giveninits=false,
    eprint=false]{biblatex}
\addbibresource{../references/bibliography.bib}


\title{Ethik im Umgang mit Daten}
\author{Name des Autors}
\date{\today}


\begin{document}

\maketitle

\abstract{
    Dies ist eine Vorlage für eine Maturarbeit in der Informatik am Gymnasium Muttenz. Sie dient dazu, die Arbeit schnell und einfach zu starten und sollte einen guten Überblick über die Arbeit bieten.
}

\tableofcontents

\chapter{Einleitung}

KI ist ein Teilgebiet der Informatik. Die Definitionen für KI sind noch recht vage, da sich das Forschungsgebiet der KI schnell und ständig weiterentwickelt. Aber die weit verbreitete ist,
dass die KI die Art und Weise wie das menschliche Gehirn Informationen verarbeitet nachahmt. KI steht für künstliche Intelligenz. 

In diesem Text geht es darum zu verstehen was KI ist, wie sie funktioniert und ein paar Auswirkungen auf unser Leben wahrzunehmen.

\chapter{Wie funktioniert KI?}

\section{Training}
KI wird programmiert um menschliche Aufgaben zu übernehmen.
KI funtioniert so, dass sie rohe Daten verwendet (bzw. Menschen stellen es der KI zur Verfügung), um sie in brauchbare Ergebnisse umzuwandeln.
Dafür programmiert der Mensch Menschenähnliche Algorythmen, die die KI immer durchläuft um versteckte Informationsmuster in Daten zu erkennen.
Je mehr KI genutzt wird desto besser wird sie. Da sie durch Erfolg und Misserfolg lernt, trainiert man sie wenn man sie benutzt.\url{https://de.wix.com/blog/beitrag/kuenstliche-intelligenz"}

\section{ Maschinelles Lernen}


Diese Algorythmen können lernen sich zu verbessern indem sie analysieren wie diese Daten wiedergegeben sind. Man nennt dies \textbf{maschinelles Lernen}.
Maschinelles Lernen unterschiedet man zwischen \textcolor{blue}{überwachtem}, unüberwachtem und bestärkendem Lernen. Bestärkendes Lernen braucht im Vergleich zu
überwachtem Lernen kein Ausgangsdatenmaterial um trainiert zu werden. Es funtioniert auf einer Trial-and-Error-Verfahrens Basis. Dabei macht die KI
eine Aktion und wird abhängig von der Auswirkung belohnt oder nicht. Ein Anwendungsbeispiel des bestärkendem Lernens ist das Alpha Go Zero von Google, das 
sich mit Weltmeistern des Brettspiels Go messen kann und sich das Spielen selber beibringen kann.
\textcolor{blue}{Überwachtes Lernen} basiert auf Ausgangsdaten, die für einen Trainingsprozess verwendet werden.\citep{bigdatainsider} Dabei wird der KI beim Training Lösungen  
schon vorgegeben. Die Trainingsdaten sind also gelabelt. Wenn die KI nach dem Trainingsprozess nicht gelabelte Daten bekommt ordnet sie sie einem zuvor erlernten
Muster einer Lösung zu. Ein Beispiel für überwachtes Lernen ist Personenerkennung auf Bildern oder automatische Erkennung von Spam E-Mails.
Unüberwachtes Lernen funktioniert gleich wie überwachtes Lernen, ausser dass keine Labels (also vorgegebene Lösungen) beim Trainingsprozess verwendet werden.



Etwas mit Änderung hier am Ende.

Wenn ich eine Quelle zitieren möchte, kann ich das ganze einfach am Ende des Satzes machen \citep{example}. Oder wie \citet{example} sagt, auch mitten im Text.

\printbibliography

\end{document}
