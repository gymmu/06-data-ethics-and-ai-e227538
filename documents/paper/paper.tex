\documentclass{report}

\usepackage[ngerman]{babel}
\usepackage[utf8]{inputenc}
\usepackage[T1]{fontenc}
\usepackage{xcolor} %Farben
\usepackage{hyperref}
\usepackage{csquotes}
\usepackage[a4paper]{geometry}

\usepackage[
    backend=biber,
    style=apa,
    sortlocale=de_DE,
    natbib=true,
    url=false,
    doi=false,
    sortcites=true,
    sorting=nyt,
    isbn=false,
    hyperref=true,
    backref=false,
    giveninits=false,
    eprint=false]{biblatex}
\addbibresource{../references/bibliography.bib}


\title{Künstliche Intelligenz }
\author{Sara Broch Jahandar}
\date{31.05.2024}


\begin{document}

\maketitle



\tableofcontents
\parindent=0em
\parskip=0.5em
\chapter{Einleitung}

KI ist ein Teilgebiet der Informatik. Die Definitionen für KI sind noch recht vage, da sich das Forschungsgebiet der KI schnell und ständig weiterentwickelt. Aber die weit verbreitete ist,
dass die KI die Art und Weise wie das menschliche Gehirn Informationen verarbeitet nachahmt. KI steht für künstliche Intelligenz. 

In diesem Text geht es darum zu verstehen was KI ist, wie sie funktioniert und ein paar Auswirkungen auf unser Leben wahrzunehmen.
\chapter{KI ist sie sie intelligent?}

\section {Intelligenz}
Wir Menschen sind fähig aus Erfahrungen zu lernen. Wir können unzählige Leistung erbringen. Wir können schwierige Probleme lösen und Hindernisse umgehen. Wir können uns an Umweltveränderungen anpassen und auf unvorhergesehene Ereignisse reagieren. Wir haben eine Wahrnehmungs- und Auffassungsgabe, die KI nicht besitzt. Die Künstliche Intelligenz die es gibt hat keine Intelligenz wie wir sie bei Menschen und Tieren beobachten können.
Es gibt zwei unterteilungen von künstlichen Intelligenzen, wie im nächsten Abschnitt beschrieben.

\section{starke und schwache KI}
Die \underline{schwache KI} auch enge KI genannt ist auf eine spezifische Aufgabe trainiert. So kann eine Spam-Erkennungs KI keine Texte korrigieren.
Beispiele für schwache KI sind \begin{itemize}
    \item Individuelle Werbeanzeige
    \item Spracherkenung
    \item Programme zur Zeichen- und Texterkennung
    \item Programme für die Erkennung von Spam E-Maschinelles
\end{itemize}

Die \underline{starke KI} hingegen ist eine hypotetische Maschine, die die Menschliche Intelligenz ganz nachahmen könnte. Die starke KI ist die künstliche Intelligenz, die wir oft in Science-Fiction-Geschichten antreffen. Diese KI hat ein logisches Denkvermögen, Emotionen und Kreativität. Wir stellen uns sie wie nachgestellte Menschen vor. 
Diese Technologien sind noch nicht entwickelt.

\chapter{Wie funktioniert KI?}

\section{Training}
KI wird programmiert um menschliche Aufgaben zu übernehmen.
KI funtioniert so, dass sie rohe Daten verwendet (bzw. Menschen stellen es der KI zur Verfügung), um sie in brauchbare Ergebnisse umzuwandeln.
Dafür programmiert der Mensch Menschenähnliche Algorythmen, die die KI immer durchläuft um versteckte Informationsmuster in Daten zu erkennen.
Je mehr KI genutzt wird desto besser wird sie. Da sie durch Erfolg und Misserfolg lernt, trainiert man sie wenn man sie benutzt.\url{https://de.wix.com/blog/beitrag/kuenstliche-intelligenz"}

\section{Maschinelles Lernen}

Diese Algorythmen können lernen sich zu verbessern indem sie analysieren wie diese Daten wiedergegeben sind. Man nennt dies \textbf{maschinelles Lernen}.
Maschinelles Lernen unterschiedet man zwischen \textcolor{blue}{überwachtem}, \textcolor{green}{unüberwachtem} und \textcolor{yellow}{bestärkendem} Lernen. \textcolor{yellow}{Bestärkendes} Lernen braucht im Vergleich zu
überwachtem Lernen kein Ausgangsdatenmaterial um trainiert zu werden. Es funtioniert auf einer Trial-and-Error-Verfahrens Basis. Dabei macht die KI
eine Aktion und wird abhängig von der Auswirkung belohnt oder nicht. Ein Anwendungsbeispiel des bestärkendem Lernens ist das Alpha Go Zero von Google, das 
sich mit Weltmeistern des Brettspiels Go messen kann und sich das Spielen selber beibringen kann.
\textcolor{blue}{Überwachtes Lernen} basiert auf Ausgangsdaten, die für einen Trainingsprozess verwendet werden.\citep{bigdatainsider} Dabei wird der KI beim Training Lösungen  
schon vorgegeben. Die Trainingsdaten sind also gelabelt. Wenn die KI nach dem Trainingsprozess nicht gelabelte Daten bekommt ordnet sie sie einem zuvor erlernten
Muster einer Lösung zu. Ein Beispiel für überwachtes Lernen ist Personenerkennung auf Bildern oder automatische Erkennung von Spam E-Mails.
\textcolor{green}{Unüberwachtes} Lernen funktioniert gleich wie überwachtes Lernen, ausser dass keine Labels (also vorgegebene Lösungen) beim Trainingsprozess verwendet werden.



\nocite{*}
\printbibliography

\end{document}
